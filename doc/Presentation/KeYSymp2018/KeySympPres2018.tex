%\documentclass{beamer}
%\documentclass[c]{beamer}
 \documentclass[t]{beamer}
%\documentclass[b]{beamer}
\listfiles

\mode<presentation>
{
% \usetheme[english]{KIT}
% \usetheme[usefoot]{KIT}
  \usetheme[deutsch]{KIT}

%%  \usefonttheme{structurebold}

  \setbeamercovered{transparent}

  %\setbeamertemplate{enumerate items}[circle]
  \setbeamertemplate{enumerate items}[ball]
}

\usepackage{babel}
\date{\today}
%\DateText

%\KITfoot{\parbox[t]{90mm}{\today:\qquad Dies ist eine sehr lange selbstdefinierte Fu\ss{}zeile -- Dies ist eine sehr lange selbstdefinierte Fu\ss{}zeile -- Dies ist eine sehr lange selbstdefinierte Fu\ss{}zeile}}


\usepackage[latin1]{inputenc}
\usepackage[TS1,T1]{fontenc}
\usepackage{array}
\usepackage{lipsum}

\usenavigationsymbols
%\usenavigationsymbols[sfHhdb]
%\usenavigationsymbols[sfhHb]

\title{Demo: AlgoVer}

\author{Sarah Grebing, Mattias Ulbrich, Jonas Klamroth}

\institute{Institut f�r Theoretische Informatik}

%\TitleImage[viewport=0 0 830 245.8,clip,height=\titleimageht]{../../bildwand.jpg}

\begin{document}

\begin{frame}
  \maketitle
\end{frame}

\begin{frame}[c]{Interaction in Interactive Program Verification}
Interaction on:
\begin{itemize}
	\item different levels of abstraction for interaction
	\item different representations of the same problem
\end{itemize}
\bigskip
Switch between levels and/or representations is necessary.
\end{frame}

\begin{frame}[c]{Involved Entities in Interactive Program Verification}
\begin{itemize}
\item program code
\item specification
\item proof representation/proof obligation
\item \emph{proof guidance/interaction}

\end{itemize}

\end{frame}

\begin{frame}[c]{Problems with Interaction in State-of-the-Art Systems}
\begin{itemize}
	\item interaction on different representations
	\item hidden dependencies between representations
	\item context change cognitively challenging for the user 
	%   \item interaction on different abstraction-levels 
	\item missing interaction possibilities on representations
\end{itemize}

\end{frame}


% 
% Ein Tool, um Interaktionskonzepte umzusetzen und zu erproben, welche
% einen m�glichst nahtlosen �bergang zwischen den Sichten erm�glichen und
% gleichzeitig alle Sichten integrieren.
% 

\begin{frame}[c]{Goal of our concept}
An interactive program verification system that allows implementing and researching different interaction concepts:
\begin{itemize}
\item integration of different representations as views
\item integration of different interaction concepts
\item seamless transition between views 

\end{itemize}

\end{frame}


% Folie 5:
% "SunBurst Diagramm":
% Hypothese: Nutzerinteraktion ben�tigt, je nach Kontext, a) einen
% fokussierten Blick auf bestimmte Elemente oder b) einen �berblick �ber
% einen gr��eren Zusammenhang.
% 
% => Daf�r ist ein Kontextwechsel notwendig
% => dieser erfordert mentale Kapazit�ten
% => Ziel ist es dabei die Last f�r den Nutzer zu reduzieren
% mit a) Anzahl der Elemente zu reduzieren und b) den Aufwand einzelner
% Wechsel zu verringern

\begin{frame}[c]{Objectives}
The user is ...
\begin{enumerate}
\item ... able to use appropriate view at all times
\item ... can easily switch views without loosing focus
\item ... is able to determine the results of costly actions before executing them  
\end{enumerate}

\end{frame}

\end{document}

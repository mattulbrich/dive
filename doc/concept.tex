\documentclass{article}

\usepackage[margin=4cm]{geometry}
\usepackage{tabularx}

\newcommand{\plus}{\textbf{+}}
\newcommand{\minus}{\textbf{--}}
\newcommand{\undecided}{\boldmath$\circ$}

\begin{document}

\title{Concept for Program-Focused Interaction in \\Algorithm Verification}
\author{MU, SG}
\date{June 2016}

\maketitle

\begin{abstract}
  We propose an interaction concept for deductive program verification
  that embraces the three main existing paradigms of interactive
  deduction: interaction, scripts and autoaction.
\end{abstract}

Algorithm (program) verification is difficult. Modern automatic
provers require guidance/assistance by the user provided in different
possible ways:
\begin{enumerate}
\item by point-and-click \emph{interaction} (alla KeY)
\item by proof menu\emph{scripts} (alla Coq, Isabelle, PVS, \dots)
\item by annotations in the program (\emph{autoactive}, alla Dafny, VCC, \ldots)
\end{enumerate}

Each of these prover guidance concepts have their advantages and
disadvantages as far the expressiveness and usability is concerned:

\begin{description}
\item[Interaction] Having a calculus for the (program) logic and rules
  that can be applied in it, gives the user full control over the
  verification process. Nothing\footnote{Nothing within Peano
    arithmetic, that is} is impossible if the calculus is relatively
  complete. Every proof can be conducted -- but it may require a
  million clicks by the user. Additionally the user must be familiar
  with the logical encoding of the program language which can differ
  considerably:\\
  \begin{tabularx}{\linewidth}{cX}
    \plus & Every proof can be conducted (manually) \\
    \minus & Logical encoding may differ from program language \\
    \minus & Fine-grained interaction obscures central/essential user
    interaction steps during proof \\
    \minus & Conducted proofs are not resilient to changes in program
    or spec since fine-grained steps refer to details which may change
    if program/spec change.
  \end{tabularx}

\item[Scripts] Scripts also provide rules for a calculus. However, the
  steps made in it are more coarse-grained and usually comprise
  ``big'' decisions like ``use structural induction with hyphothesis
  $\phi$ here'' or ``make case distinction on $\psi$'' or ``solve this
  with special solver x''. Often the original proof goal is broken
  down into intermediate targets which are then easier to reach. Some
  scripts reuqire the user to formulate proofs constructively, in a
  forward manner. This may make them more readable.\\
  \begin{tabularx}{\linewidth}{cX}
    \plus & Proofs can be arranged in a (relatively)
    readable/comprehensible manner\\
    \plus & Conducted proofs can be made relatively resilient to
    changes in program or spec since, in the coarse steps little
    reference to details need to be done.\footnote{They are, however,
      susceptible to changes in the theories, as experience shows}\\
    \minus & Logical encoding may differ from program language\\
    \undecided & Detailed insight into logic may not be possible.
  \end{tabularx}

\item[Autoaction] All communication with the prover is on the level of
  annotations to the program. There is no visible logical layer behind
  it. This allows the prover to encode things more efficiently since
  (human) comprehensibility is no issue. Likewise, the user has not to
  keep two mental models (program/logic) synchronised in their
  minds. Understanding why a proof fails becomes a challenge then, and
  the user is limited in their means to give hints to the prover:
  Usually only a few built-in tools like case distinctions, local
  lemmas, induction are available.\\
  \begin{tabularx}{\linewidth}{cX}
    \plus & There is no ``second (logical) level''. User can (in
    theory) think in terms of programs only. \\
    \plus & Annotations are relatively spares and the rationale behind
    proofs can be made comprehensible\\
    \minus & Programming constructs like if-then-else or assertions
    are used for prover control; additional fake predicates may be
    needed which may make proofs incomprehensible again. \\
    \minus & Insight into the logical level is missing entirely.
  \end{tabularx}
\end{description}

In this project we propose an interaction concept which brings
together the advantages of the three paradigms; in particular it
allows for interaction by all three means in a coordinated manner.

\begin{enumerate}
\item The interaction concept comprises different views on the
  verification problem:
  \begin{enumerate}
  \item code/algorithm view (p-view)
  \item proof obligation overview (po-view)
  \item logical encoding (l-view)
  \item proof manuscript (s-view)
  \end{enumerate}

\item The views are highly connected in the sense that entities which
  belong together can be highlighted, allow navigation, etc.  In the
  logical view, every statement has references to its
  predecessor/origin/purpose/code location.

\item The logical encoding is on a logic which is as identical as
  possible to the programming language.

  The logical representation is always kept as closely as possible to
  artifacts that occur in the program. No normalisation etc.\ takes
  place.

\item Proof progression is possible on all views.

  The relevant means for the progression need to be investigated: Only
  a few builtin rules with good support or user-provided rules?

\item Proof progression made in one view is reflected in the others.
  An interactive proof step modifies the proof script. An autoactive
  command is reflected as a new proof obligation, etc.
\end{enumerate}

Not only conducting the proof is difficult. Proof comprehension is
also a hard business. Therefore the concept will provide different
tools to cope with the challenge:
\begin{enumerate}
\item \emph{Abstraction} According to a user-given criterion, abstract
  away from concrete entities in logical encoding. Use \emph{pre}
  instead of the actual precondition, etc.
\item \emph{Selection} According to a user-given criterion (a set of
  variables, set of heap locations, line in program, ``everything that
  has to do with arithmetic'', etc.), only parts of the proof
  obligation, program, and script are displayed.
\item \emph{Assumption} This allows the user to dive into ``what-if''
  scenarios by making additional assumptions as conjectures which can
  be taken back easily (backtracking).
\end{enumerate}
These tools are meant for inspection of the proof state, less for
progress (assumption may make progress).

\paragraph{Goals and Objectives}
In this project a new interaction concept will be developed which addresses 
common problems in performing software verification tasks:

\begin{itemize}
 \item Finding the cause of a failed verification attempt, i.e., either the 
prover is not able to find a proof or an error in the specification or program 
was introduced by the user
\item where, in terms of the program, did the verification process stop
\item fast recovery from verification errors 
\end{itemize}

To address these problems our new interaction concept will allow different 
presentations of the proof problem as well as new actions during the proof 
process.

The concept allows for a seamless phase shift between autoactive and 
interactive proof concepts and allow three different projections of the proof 
state into the three XXXdomains.

    	
%    Obseravtions What are users actually doing? As opposed to what you 
% expected they might % do.
%     What routines do users have with the product? How are they integrating it 
% into their lives?
%     Record details – adding granularity and specificity to an observation can 
% make it much more meaningful
%     Ensure you’re examining activities in their whole; look at how the product 
% is used in context with their device and the flow of their lives and not just at 
% the product itself.
%     Don’t be afraid to get qualitative. If you see an example of behavior that 
% you think may be repeated – make a note of it and look for it in future 
% observations.


\section{Use Cases from User Study}
\subsection{Proof Process}
User writes program and first specification. It might be the case that a user 
writes the full specification at once, however we assume the develops the 
specification during the process.




We conducted a user study using contextual interview methods with nine users 
from the KeY system (interactive, with explicit proof object, rule applications 
via point and click)
to extract issues when proving a problem interactively.

We provided three use cases and the users added XX use cases.

As actor in all use cases serves the standard user: a developer or a researcher 
(both with background in theorem proving/verification)
In the following the use cases and their preconditions are depicted. 

\subsection{Use Case: Find/search cause of open proof}

Flow-of-Events for this use case:
\paragraph{1. Preconditions}
\begin{itemize}
 \item user has written (first) specification
 \item user has invoked systems automatic strategies to find a proof
 \item system didn't find a proof and displays open proof to user
\end{itemize}



\subsection{Use case: develop specification}
Flow-of-Events for this use case:
\paragraph{1. Preconditions}
\begin{itemize}
 \item user has written (first) specification
 \item user has invoked systems automatic strategies to find a proof for first 
specification
 \item system found a proof 
 \item system didn't proof problems of whole system under proof yet
\end{itemize}

\subsection{Use case: Perform Proof Attempt}
Flow-of-Events for this use case:
\paragraph{1. Preconditions}
\begin{itemize}
 \item user has written (first) specification
\end{itemize}
\subsection{Use case: Continue/Resume a proof}
Flow-of-Events for this use case:
\paragraph{1. Preconditions}
\begin{itemize}
 \item user has written (first) specification
 \item user has invoked systems automatic strategies to find a proof
 \item system didn't find a proof and displays open proof to user
 \item user has found cause for failed proof attempt
 \item user corrected error
\end{itemize}
\subsection{Use case: Correct error}
Flow-of-Events for this use case:
\paragraph{1. Preconditions}
\begin{itemize}
 \item user has written (first) specification
  \item user has invoked systems automatic strategies to find a proof
 \item system didn't find a proof and displays open proof to user
\end{itemize}
\subsection{Use case: write specification}
Flow-of-Events for this use case:
\paragraph{1. Preconditions}
\begin{itemize}
 \item user has written program 
 \item user has read natural language spec
\end{itemize}
\section{Usability heuristics will be followed}

The new concept will follow the following general heuristics by Nielsen for UI 
design:
\begin{enumerate}
 \item Match between system and real world
 \item Recognition rather than recall
 \item minimalist design
 \item help user recognize, diagnose and recover from errors
 \item visibility of system status
 \item MAYBE: flexibility and efficiency of use; user control and freedom, help 
and documentation; error prevention
\end{enumerate}

These heuristics for user interface design (by Nielsen) will be stated more 
precisely in the context of program verification systems in the following.

\paragraph{Match between system and real world}
\begin{quote}
 The system should speak the users' language, with words, phrases and concepts 
familiar to the user, rather than system-oriented terms. Follow real-world 
conventions, making information appear in a natural and logical order.
\end{quote}

When adapting the general heuristic stated by Nielsen to verification systems 
the user's input language is the program and its specification. Our hypothesis 
is, that most of the time in the verification process the user thinks in 
terms of the program and the specification that should hold in specific 
program states. This implies, that users may think in terms of program states 
and program execution traces. Therefore the natural order for appearances of 
conditions, assertions and assumptions is the order they appear and 
should hold during a specific program execution trace.

First of all, the user should be able to stay as long as possible on the 
program level for proof guidance and comprehension. If this is not possible 
anymore, due to information that can't be expressed purely on the program 
level, the user should be able to switch between a logical view of the proof 
state and the program view. However, to ease the change dependencies and 
correspondences between both views should be explicitly visible, by for example 
highlighting corresponding parts/entities in both views when hovering over them 
or using same colors for same entities.

In addition path conditions and variable updates should stay in the programming 
language even if switched to the logical level. Especially, naming of entities 
should not change between both views. The latter two requirements will be 
reflected by the next heuristic.
\paragraph{Recognition rather than recall}
\begin{quote}
 Minimize the user's memory load by making objects, actions, and options 
visible. The user should not have to remember information from one part of the 
dialogue to another. Instructions for use of the system should be visible or 
easily retrievable whenever appropriate.
\end{quote}

To adapt this heuristic, we have to state what actions, objects and options are 
present when proving programs.

First of all we have the program, containing of statements, control structures, 
variables,... together with a specification containing of logical terms that 
formalize properties about structures appearing in the program.

We have program states, proof states, rule applications, strategy invocations, 
and mathematical proof constructs.
These constructs appear across the different views and have dependencies 
between each other:


\paragraph{Minimalist design}
Actually the heuristic is called Aesthetic and minimalist design stating 
\begin{quote}
 Dialogues should not contain information which is irrelevant or rarely needed. 
Every extra unit of information in a dialogue competes with the relevant units 
of information and diminishes their relative visibility.
\end{quote}

\paragraph{help user recognize, diagnose and recover from errors}
\begin{quote}
 Error messages should be expressed in plain language (no codes), precisely 
indicate the problem, and constructively suggest a solution.
\end{quote}

\paragraph{ visibility of system status}
\begin{quote}
 The system should always keep users informed about what is going on, through 
appropriate feedback within reasonable time.
\end{quote}

\paragraph{Details}
NOTE: Do not bother about language in the Enumeration below
\begin{enumerate}
\item the user always has to be kept informed about the progress of the system 
and the proof. This has to be realized by the po-view, where a matrix of all 
pvc belonging to a po have to be shown and it should be indicated in what state
 they are in. (see verifast?+why?)
\item ``easy to solve'' pvcs have to be solved automatically by invoking 
different tools (z3, dafny, key)
\item problems that can not be solved by z3 or dafny should be given to the key 
system for further inspection and proof progress
 \item The  p-view always has to be visible as it is the formulation of the 
problem by the user. Our hypothesis is, that the user thinks in most cases in 
terms of the programme 
 \item If P-view and L-view are 
visible side-by-side, the P-view has to reflect the path through the programm 
that is shown by the L-view. That is, all lines not belonging to the l-view 
have to be greyed out. if hovering over formulas in the l-view, the 
corresponding origin of the p-view has to be highlighted to reflect dependencies

\end{enumerate}


\paragraph{Stuff}

\end{document}
